\chapter{Conclusions}

In conclusion, we have developed a method that allows us to easily and quickly generate a dataset for training a neural network to perform a pose estimation task. Our approach results in good performance for a set of objects of interest in a specific environment, and does not require expensive or complicated instrumentation. We also developed a reliable approach for semantic meaning extraction used in a specific assembly task. Finally, we succesfully applied our method to a real-world scenario, and used our estimations to drive the movement of a robotic manipulator.

Possible future developments may include:
\begin{itemize}
    \item Comparing the performance of a network trained on one of our generated datasets with a network trained on real-world labelled data.
    \item Verifying the possibility of avoiding false positives with the model by training it to ignore objects.
    \item If future pose estimation networks increase performance in a significant manner, verifying if they obtain the accuracy required for high precision robotics applications.
\end{itemize}